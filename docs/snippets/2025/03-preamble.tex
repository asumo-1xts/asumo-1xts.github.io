\documentclass[uplatex, twocolumn]{jlreq} % jsarticleでも可

\usepackage[style=numeric-comp]{biblatex} % BibLaTeXパッケージ読み込み
\addbibresource{ref.bib} % bibファイルを登録
\ExecuteBibliographyOptions{ % 追加オプション(他にも色々)
  sorting = none, % 引用した順に並べる
  maxnames = 2, % 連名になっている著者の数がmaxnamesを超えると、
  minnames = 1 % 初めのminnames人だけ表記されて残りは省略される
}

% ここまでは常套手段、検索結果も多数
% ===============================================================================
% ここから奥の手

\usepackage[british, english]{babel}
% babelパッケージにjapaneseは無いらしい
% 以下のDefineBiblio~コマンドを使いたいけれど言語設定が必要なので、
% 便宜的にjapaneseの代わりにbritishを割り当てておく
\DefineBibliographyStrings{british}{andothers={他}} % 和文の文献なら「他」
\DefineBibliographyStrings{english}{andothers={\textit{et al.}}} % 斜体にする

\AtEveryBibitem{ % bibファイルの文献を走査してゆく
  \iffieldequalstr{langid}{Japanese}{ % if: langid=Japaneseとした文献のみピックアップ
    \selectlanguage{british} % 言語をbritishにすることでandothers={他}が適用される
    \DeclareDelimFormat{finalnamedelim}{ % 区切り文字の設定
      \ifnumgreater{\value{liststop}}{2}{\finalandcomma}{} % 「and」を使わせない
    \addspace\multinamedelim }
    \DeclareFieldFormat{title}{\textrm{「#1」}} % 論文のタイトルを普通の字体で鍵括弧囲み
    \DeclareFieldFormat[book]{title}{\textrm{『#1』}} % 書籍なら二重鍵括弧
    \DeclareFieldFormat{journaltitle}{\textrm{#1}} % 雑誌のタイトルも忘れず斜体解除
    \DeclareFieldFormat{booktitle}{\textrm{#1}} % 書籍のタイトルも忘れず斜体解除
    \renewbibmacro{in:}{} % ジャーナル名の前の「In:」を除去
  }{% ここまでが和文の文献に限った処理
  \selectlanguage{english}} % else: 欧文の文献はデフォルト(english)の処理で良い
}

% 奥の手ここまで(プリアンブル終了)